\documentclass[12pt,letterpaper]{amsart}
\setlength{\oddsidemargin}{.0in}
\setlength{\evensidemargin}{.0in}
\setlength{\textwidth}{6.5in}
\setlength{\topmargin}{-.3in}
\setlength{\headsep}{.20in}
\setlength{\textheight}{9.in}
\usepackage[leqno]{amsmath}
\usepackage{amsfonts}
\usepackage{amssymb}
\usepackage{amsthm}
\usepackage{amssymb}
\usepackage[all]{xy}
\usepackage{graphicx}


%Here are some user-defined notations
\newcommand{\RR}{\mathbf R}
\newcommand{\CC}{\mathbf C}
\newcommand{\ZZ}{\mathbf Z}
\newcommand{\ZZn}[1]{\ZZ/{#1}\ZZ}
\newcommand{\QQ}{\mathbf Q}
\newcommand{\rr}{\mathbb R}
\newcommand{\cc}{\mathbb C}
\newcommand{\zz}{\mathbb Z}
\newcommand{\zzn}[1]{\zz/{#1}\zz}
\newcommand{\qq}{\mathbb Q}
\newcommand{\calM}{\mathcal M}
\newcommand{\latex}{\LaTeX}
\newcommand{\tex}{\TeX}
\newcommand{\sm}{\setminus} 


%improving spacing in tables (space above and below characters in a row)
\newcommand{\tfix}{\rule{0pt}{2.6ex}}
\newcommand{\bfix}{\rule[-1.2ex]{0pt}{0pt}}



%Here are commands with variable inputs 
\newcommand{\intf}[1]{\int_a^b{#1}\,dx}
\newcommand{\intfb}[3]{\int_{#1}^{#2}{#3}\,dx}
\newcommand{\marginalfootnote}[1]{%
        \footnote{#1}
        \marginpar[\hfill{\sf\thefootnote}]{{\sf\thefootnote}}}
\newcommand{\edit}[1]{\marginalfootnote{#1}}


%Here are some user-defined operators
\newcommand{\Tr}{\operatorname {Tr}}
\newcommand{\GL}{\operatorname {GL}}
\newcommand{\SL}{\operatorname {SL}}
\newcommand{\Prob}{\operatorname {Prob}}
\newcommand{\re}{\operatorname {Re}}
\newcommand{\im}{\operatorname {Im}}


%These commands deal with theorem-like environments (i.e., italic)
\theoremstyle{plain}
\newtheorem{theorem}{Theorem}[section]
\newtheorem{corollary}[theorem]{Corollary}
\newtheorem{lemma}[theorem]{Lemma}
\newtheorem{conjecture}[theorem]{Conjecture}

%These deal with definition-like environments (i.e., non-italic)
\theoremstyle{definition}
\newtheorem{definition}[theorem]{Definition}
\newtheorem{example}[theorem]{Example}
\newtheorem{remark}[theorem]{Remark}

%This numbers equations by section
\numberwithin{equation}{section}


\begin{document}


\begin{titlepage}
\title{Title Here}
\author{Your Name}
\date{Date Here}
\maketitle

\centerline{\Large Math 2784 (or 2794W)}

\hfill

\centerline{\Large University of Connecticut}
\thispagestyle{empty}
\end{titlepage}
\pagebreak



\thispagestyle{empty}
\tableofcontents


\vfill



\pagebreak
 
\pagenumbering{arabic}

%%%Start your work here. 

\section{Introduction}\label{intro}


In this file, edit the information between  
\verb1\begin{titlepage}1 and \verb1\end{titlepage}1.
Do {\it not} change the typesetting commands such as 
\verb1\setlength1 at the top of the file, which 
affect the size of the output.

You might not want to use a table of  
contents in your paper, 
and in that case just 
take out the line \verb1\tableofcontents1. (Better idea: place 
a \verb1%1 at the front of the line.  This hides the line from 
the \latex{} typesetting process without removing it.)
Without a table of contents, you 
should help the reader by summarizing the organization 
of the paper at the end of your introduction.
Anyway, write your paper between  
\verb1\section{Introduction}\label{intro}1 and \verb1\end{document}1.
Hint: if you need any \latex{} construction, an equation for example, 
just find a similar one in one of the \latex{} files you have, 
copy, paste and edit.  Or ask a math professor (the younger ones 
all know \latex).

In this section you should put an introduction.  
Tell us what your topic is about, roughly, 
and what you are going to do with it. 


\section{Section 1}\label{sec1}

Make the title of this section a little bit more descriptive than 
the banal ``Section 1.'' 



\section{Section 2}\label{sec2}

Same deal as in the previous section. 


\section{Section 3}\label{sec3}

If you really need it, here is a third section.  
You can generate as many as you need. 


\appendix

\section{Appendix stuff}\label{app1sec}

Perhaps in the main part of your text there are some 
long and boring calculations 
whose appearance would interrupt the flow of ideas, or there is 
something you want to 
describe briefly but not get bogged down by further in the main 
text of your paper.  If 
further information 
is nevertheless worthwhile, consider placing it in an appendix.  
Here the numbering of equations and theorems gets an A attached to 
reflect the location:

\begin{theorem}
The function $\sin x$ is the unique solution of the differential equation
\begin{equation}\label{sineqn}
\frac{d^2y}{dx^2} + y = 0
\end{equation}
satisfying the initial conditions $y(0) = 0$ and $y'(0) = 1$.
\end{theorem}



\section{More appendix stuff}

It is doubtful that a $7\frac{1}{2}$ page paper should have a second appendix, 
but here is one so you can see it.  All labels get slapped with a B. 

\begin{conjecture}
Papers of this length are unlikely to have a second appendix.
\end{conjecture}

\begin{proof}
We argue by contradiction.  For instance, consider this paper as a base case. 
It has a second appendix.  This is a contrapositive, or is it a converse? 
Well, this proof makes no sense, but it does seem weird 
anyway to be proving a conjecture 
right after stating it.  Better luck next time.
\end{proof}


\begin{thebibliography}{5}

\bibitem{gt}
B. H. Gross and J. T. Tate, Commentary on algebra, pp.~335--336 in: 
``A Century of Mathematics in America, Part II,'' Amer. Math. Soc., 
Providence, 1989.

\bibitem{irros}
K. Ireland and M. Rosen, ``A Classical Introduction to Modern 
Number Theory,'' 2nd ed., Springer-Verlag, New York, 1990.

\bibitem{unabomber}
T. J. Kaczynski, Another proof of Wedderburn's theorem, 
{\it Amer. Math. Monthly} {\bf 71} (1964), 652--653.


\bibitem{knus}
M-A. Knus, A. Merkurjev, M. Rost, J-P. Tignol, 
``The Book of Involutions,'' Amer. Math. Society, Providence, 1998.

\bibitem{roquette}
P. Roquette, Class field theory in characteristic $p$, its origin 
and development, pp.~549--631 in: ``Class field theory -- its centenary 
and prospect,'' Math. Soc. Japan, Tokyo, 2001.

\bibitem{sloane}
N. J. A. Sloane, On-line Encyclopedia of Integer Sequences, 
{\tt http://www.research.att.com/$\sim$njas/ 
sequences/}.

\end{thebibliography}

\end{document}








